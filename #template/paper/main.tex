\documentclass[12pt]{article}
\usepackage{lipsum}
\usepackage{z_paper, z_math}


\begin{document}

% =========================
% ===== Page de garde =====
\newcommand{\HRule}{\rule{\linewidth}{0.5mm}}

\begin{titlepage}
\begin{center}

% Upper part of the page. The '~' is needed because only works if a paragraph has started.
\includegraphics[width=0.45\textwidth]{Core/logo.png}~\\[1cm]

%\textsc{\LARGE ENSAE Paris}\\[1.5cm]

\textsc{\Large }\\[0.5cm]

% Title
\begin{center}

\HRule \\[1cm]
{\huge \bfseries{Fiches d'économétrie 2}}\\[0.5cm]
\HRule \\[2.5cm]

\end{center}


% Author and supervisor
\begin{minipage}{0.4\textwidth}
\begin{flushleft} \large
\emph{Auteur:}\\
Pierre \textsc{Rouillard}\\


\end{flushleft}
\end{minipage}
\begin{minipage}{0.4\textwidth}
\begin{flushright} \large
%\emph{Référent:} \\
%Pierre-Antoine \textsc{Robert}
\end{flushright}
\end{minipage}

\vfill

% Bottom of the page
{\large \today}

\end{center}
\end{titlepage}


% =========================
% ========= TOC ===========
\pagestyle{fancy}
\fancyhead[L]{}
\fancyhead[R]{}
\thispagestyle{empty}
\tableofcontents
\break
\setcounter{page}{1}

% =========================
% ========= MISC ==========

\addcontentsline{toc}{section}{Acknowledgement}
\section*{Acknowledgement}
\lipsum[1]
\newpage
\section*{Introduction}
\lipsum[1]
\newpage


% =========================
% ======== START ==========
% > Change page style
\pagestyle{fancy}
\fancyhead[L]{\small{\leftmark}}
\fancyhead[R]{}

\section{xxx}

\lipsum[1]

\subsection{xxx}

\quad A \blue{largé} part of the Structural VAR framework analysis has to do with (orthogonal)
structural shocks identification. \redl{Several approachs} have been developed and discussed
throughout the years, such as recursive identification (Sims (1980) [1] i.e imposing zero
restrictions so that variables do not depend contemporaneously on the shocks ordered
after), short and long-run restrictions (respectively zero restrictions: on a subset of
shocks for specific variable(s) and on some coefficients of the long-run matrix). These
identification \ghl{schemes yield exact identification} in the sense that a shock is uniquely
identified through precise estimation of the matrix $B0$ (see R1). However sign restrictions
have also been discussed and in this case, we have a pool of plausible models and thus
only partial identification. All these identification procedures usually rely on economic
theory to justify restriction choices.
\bigbreak
\begin{boxH}
    \textbf{Modèle Tobit II}\par
    \quad $ Y^{*} = X'.\beta_{0} + \varepsilon $ \quad \red{distribution de $\varepsilon$ non spécifiée}\par
    \quad $Y = D.Y^{*}$ est seulement observé.
\end{boxH}

\begin{enumerate}
    \item test
        \begin{enumerate}
            \item test2

        \end{enumerate}
    \item test
    \item test
        \begin{enumerate}
            \item test2
            \item test2
        \end{enumerate}
    \item test
\end{enumerate}

\end{document}


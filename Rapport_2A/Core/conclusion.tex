\quad Throughout this work we have firstly shown that, as the Covid-19 crisis unveiled, the six studied advanced economies experienced a period of accrued savings by households. 
While this stock of excess savings remains large in European countries it has started rapidly depleting in the US as soon as late 2021. 
In addition to the disparities between countries, we argued that these savings were unevenly distributed across income groups.

We then assessed the developments of wage growth in European countries. 
Using the Autoregressive Distributed Lag framework we were able to compare different slack measures and their relative explainability power regarding wage changes.
While our model-selection methodology showed better perfomance for some alternative slack measures compared to the standard unemployment rate to explain wage developments, we still had to select the model using the latter to compute forecast. 
The selected models provided insights on wage developments since the Covid crisis and highlighted the embedded increasing share of inflation. In an attempt to explain the changes in productivity affecting wage growth we showed that labour productivity changes are mainly driven by country-specific structural factors.

Rather than becoming a technical expert on a specific topic, this internship provided me with a broader view of economic subjects and ways to conduct research within an corporate environment. 
In addition to the two projects presented in this report, I also worked on topics such as a small global trade model\cite{az_trade}; an assessment of increasing inflation’s effect on sovereign debt ratio (materials for Allianz Trade CEO\cite{az_debt}); or computing corporate profit margins and introducing a proxy for some industries\cite{az_margins}.
\quad Over the past year I first spent six months at AEW (Europe) within the Paris team of the Research \& Strategy department from June to November 2022. 
AEW represents the real estate asset management platform of Natixis Investment Managers. 
The Paris team has three permanent members (two directors Mr. Baccam and Ms. Fossé and one data analyst Mr. Mejri) and usually has one or two interns. 
The range of work the team is involved in is really wide, even for the interns, whether it is for solely in-house purposes (for example for other teams) or to be published research works. 
During my time in the team I had the chance to be involved in working on at least one of every publication type, which I am going to explain for better understanding. 

First, the team produces a monthly research report on (usually) high profile, or more forward looking, topics regarding the european real estate markets (office, logistics, retail or housing). 
These publications are then published on the company website and sent to partners. 
It is public available information that also serve as advertisement to showcase the company’s work. 
Second are in-house reports and this is mainly what interns work on (mostly in English). 
Most of these reports are requested by the different investment funds within the company. 
It usually consists in a concise slideshow with comments and forecast on the side. 
The funds are often speciliased in investment towards specific markets, for instance I worked a lot with Logistis which focuses on the French logistics market. 
Thus, the requests are mainly to update the \textit{‘research note’} associated to the fund who is calling for an update. 
In the latter we find an economic outlook that screens the economy (country-specific or a region depending on the fund) and provides a broad summary of the past quarter economic developments (inflation, GDP growth, interest rates, labour market etc…). 
Then usually follow an Occupier market and an Investment market part. 
The first one essentially focuses on the real market variables recent trends (take-up, vacancy rate, rents…) and the second on market/financial variables (yields and investment volumes). 
This is what we call \textit{Market Update Reports} and they are not really long publications but it is more the number of funds requesting these quarterly updates around the same dates that make it a lot of work.

I think this intership was the best way to start my gap year as I was a bit lost, had no particular field that I was interested in and just broadly felt that I lacked experience in working on economics related subjects. 
Chosing a strongly specialied research department such as the one of AEW focusing on real estate allowed me to first: strengthen my knwoledge of basic economics subjects and make concrete links between them when I had to draft the outlook parts of the reports, but it also introduced me to a field that I surprisingly developed a strong interest in. 
I think the way the team works played a key role in the latter. 
Indeed, the interns are not just involved in drafting parts of the Market Update Reports but the whole thing in itself. 
At the beginning of the quarter we are handed with requests from the different funds and the work is divided between the interns. 
It is rewarding and inclusive as we have our own agenda and can organise it as we feel, as look as we meet the deadlines. 
The team being relatively small and enthusiastic also makes it increasingly motivating. 
Moreover, the work is not thoughtless charts updates as we go through reports, news and talks to draft our analysis and the reviews from my supervisor were always enriching. 
I was able to draft better, more concise and relevent comments as time went on thanks to his advice. 
The interns are also tasked with writing the \textit{Quarterly Situation Report} for the French and European real estate markets. 
These two reports are (outside of client presentations) the longest in-house materials we work on and consist in a global overview of recent developments across the real estate sectors. 
Regarding the European note we do not go over every country but rather focus on the countries the French side of the company is heavily invested in: France, Germany, Belgium, Spain and the Netherlands. 
This is probably the reports I enjoyed working on the most as it required summarising a lot of information, while remaining clear and easy to go through which is something I struggled to do at the beginning but improved over time. 
The wide range of topics adressed in these reports also made it all the more interesting. 
Regarding the actual work I was involved in, I went through two quarters of udpates and reporting. Below is the list of projects I worked on.

\begin{itemize}
    \item French and European Q2 and Q3 (2022) Quarterly Situation Report that I previously mentioned during a particularly interesting period with rising interest rates, questioning from investors and increasing tension on the markets.
    \item Over the summer during one of our colleagues’ break from London I took over a request from the \textit{Halog} fund. The report consists in an economic outlook for the eurozone as of Q2 2022. It was around July/August 2022 so it remained uncertain despite earlier recovery from the pandemic, mainly due to the conflict in Ukraine. In addition to the increases in interest rates, real GDP growth for the eurozone had been revised downwards by 20 bps compared to the first quarter projections. The reports goes on with a property market overview for the eurozone logistics market and then country-specific market updates (France, Germany, the Netherlands, Belgium, Dernmark and Austria). This is the first report where I actually suggested charts improvement on my own and tried to make most of the training period I went through in the prior weeks. Despite tight schedule I am thankful for the team manager Mr. Vrensen supervising my work and giving me a better understanding of economic variables I was struggling with, especially when it came to how yields and prime yields were determined. The report was also intersting as I had to do some research on Danish and Austrian markets which I had never worked on. 
    \item Then came the updates for four different funds that I took over from the other intern I replaced. Namely, \textit{Logistis}, \textit{PBW1}, \textit{Fondis}, \textit{Celsius} and \textit{Ulis}. I already mentioned \textit{Logistis} before and to go over the other three \textit{PBW1} focuses on investment in Poland and Czech Republic (office and retail markets), \textit{Fondis} on the French retail market, \textit{Celsisus} on the French and Belgian retail markets and \textit{Ulis} on different european logistics markets. It was challenging but educative as I had to manage my agenda to provide reports in time, meet different teams’ needs and improve in my writing skills when it came to delivering clear messages in a few points. The reports format were pretty similar with first a global outlook, then some property market slides and comments, then a focus on the investment side. They were especially concerned with rising interest rates leading to increasing funding costs and yields, thus potentially decreasing property  values.
    \item The \textit{Quarterly Real Estate Market Update} of Septembre 2022 for France. It is basically a longer and more in depth version of the French \textit{Quarterly Situation Report} that incorporates more details on different French regional markets. It is intended to AEW Patrimoine’s clients (basically the real-estate investment trust (REITs) side of AEW -‘SCPI’ in French-).
    \item A global market update for Germany in Q3 2022, that goes over each sector (logisitcs, retail, office, housing) and showcases data for the main German cities. 
\end{itemize}





% === ANGLAIS ===
\quad During the second part of my gap year, I interned at Allianz Trade within the Economic Research department of the Paris team. 
Following my internship at AEW I was interested in joining a research department where I could be involved in publication work on a wide range of economic subjects. 
In the six months I spent within the team I had the chance to work on many different releases, and I decided to focus on two specific ones.

The first part of my report sheds light on several critical aspects of the stock of household excess savings in both European countries and the United States. 
First and foremost, it becomes evident that the stock of excess savings continues to be substantial in absolute terms in these regions. 
European countries have only recently begun to experience a slight reduction in this surplus, while American households embarked on the depletion of their savings as early as Q4 2021.

Moreover, the distribution of excess savings among income groups highlights a significant disparity. 
As of early 2023, the bottom 40\% of the income spectrum in these regions have fully depleted their excess savings, underscoring the unequal impact of the prevailing economic conditions. 
Conversely, higher income households tend to view their excess savings as assets rather than immediate disposable income, resulting in a reluctance to spend, which has implications for overall economic recovery.

Furthermore, the difference in consumption patterns between the United States and European countries is noteworthy. 
Following the pandemic, the United States experienced a more rapid increase in consumption spending compared to European nations. 
However, this heightened consumption has led to a swift depletion of excess savings in the US. 
The increasing financial strain on American households indicates that this trend is likely to persist, potentially resulting in the complete exhaustion of excess savings throughout the year.

In light of these findings, we expect policymakers and financial institutions to closely monitor the dynamics of excess savings, as they play a pivotal role in shaping both short-term economic recovery. 
Addressing income disparities various income groups will be paramount in achieving a balanced and sustainable economic outlook. 
Additionally, a vigilant approach to managing the rapid depletion of excess savings in the United States will be crucial to ensure the resilience of households in the face of economic challenges.

The second project presented in this report shows that various alternative measures of labour market slack can perform equally well, if not better, than the traditional unemployment rate in explaining wage growth. 
These alternative metrics consistently reveal disparities in inflationary pressures on wages between peripheral and core countries within the Eurozone.

Over the past year, wage growth has exhibited a notable strengthening, buoyed by robust labour markets and efforts to compensate workers for surging inflation. 
However, it is anticipated that this upward trajectory in wage growth may gradually decelerate, especially as inflation subsides and economic activity slows in core countries. 
On the contrary, peripheral countries may confront a different challenge, potentially experiencing delayed but rapid wage increases that could exert additional pressure on core inflation.

Additionally, it is crucial to acknowledge the diverse dynamics at play within the Eurozone concerning changes in productivity. 
These variations are largely attributable to structural factors, further underscoring the need for a nuanced approach in addressing productivity disparities among member states.

% === FRANCAIS ===
Lors de la deuxième partie de mon année de césure, j'ai effectué un stage chez Allianz Trade au sein du département de recherche économique de l'équipe parisienne. 
Après mon stage chez AEW, je souhaitais rejoindre un département de recherche où je pourrais participer à des travaux de publication sur un large éventail de sujets économiques. 
Au cours des six mois que j'ai passés au sein de l'équipe, j'ai eu l'occasion de travailler sur de nombreuses publications différentes, et j'ai décidé de me concentrer sur deux d'entre elles en particulier.

La première partie de mon rapport souligne plusieurs aspects essentiels du stock d'épargne excédentaire des ménages dans les pays européens et aux États-Unis. 
Tout d'abord, il apparaît clairement que ce stock d'épargne reste conséquent en termes absolus dans ces régions. 
Les pays européens n'ont commencé que récemment à enregistrer une légère réduction de cet excédent, tandis que les ménages américains ont commencé à réduire leur épargne dès le quatrième trimestre 2021.

En outre, la répartition de l'épargne excédentaire entre les strates de la population met en évidence une disparité importante. 
Début de 2023, les strate gagnant entre 0 et 40\% des revenus les plus bas dans ces régions ont entièrement épuisé leur épargne excédentaire, ce qui souligne l'impact inégal des conditions économiques actuelles. 
À l'inverse, les ménages aux revenus plus élevés ont tendance à considérer leur épargne excédentaire comme un actif plutôt que comme un revenu disponible immédiat, ce qui se traduit par une réticence à dépenser, avec les conséquences que cela implique pour la reprise économique globale.

La différence entre les modèles de consommation des États-Unis et des pays européens est également remarquable. 
Après la pandémie, les États-Unis ont connu une augmentation plus rapide des dépenses de consommation que les pays européens. 
Cependant, cette consommation accrue a conduit à un épuisement rapide de l'épargne excédentaire aux États-Unis. 
La pression croissante pesant sur les ménages américains indique que cette tendance est susceptible de persister, ce qui pourrait entraîner l'épuisement complet de l'épargne excédentaire avant la fin de l'année 2023.

À la lumière de ces résultats, nous nous attendons à ce que les décideurs politiques et les institutions financières surveillent de près la dynamique de l'épargne excédentaire, car elle joue un rôle essentiel dans l'élaboration de la reprise économique à court terme. 
Il sera essentiel de s'attaquer aux disparités de revenus entre les différentes strates de la population pour parvenir à des perspectives économiques équilibrées et durables. 
En outre, une approche vigilante relative à l'épuisement rapide de l'épargne excédentaire aux États-Unis sera cruciale pour assurer la résilience des ménages face aux défis économiques.

Le deuxième projet présenté dans ce rapport montre que diverses mesures alternatives de tensions sur le marché du travail peuvent expliquer la croissance des salaires aussi bien, voire mieux, que le taux de chômage traditionnellement utilisé. 
Ces mesures alternatives révèlent systématiquement des disparités dans les pressions inflationnistes sur les salaires entre les pays périphériques et les pays \textit{core} de la zone euro.

Au cours de l'année écoulée, la croissance des salaires s'est sensiblement renforcée, grâce à la robustesse des marchés du travail et aux efforts déployés pour compenser l'inflation croissante. 
Toutefois, on s'attend à ce que cette trajectoire ascendante de la croissance des salaires ralentisse progressivement, en particulier à mesure que l'inflation s'atténue et que l'activité économique ralentit dans les pays \textit{core}. 
Au contraire, les pays périphériques pourraient être confrontés à un défi différent, avec des augmentations salariales retardées mais rapides qui pourraient exercer une pression supplémentaire sur l'inflation sous-jacente.

Il est également essentiel de reconnaître les diverses dynamiques à l'œuvre au sein de la zone euro en ce qui concerne les changements de productivité. 
Ces variations sont largement attribuables à des facteurs structurels, ce qui souligne la nécessité d'une approche nuancée pour remédier aux disparités de productivité entre les États membres.









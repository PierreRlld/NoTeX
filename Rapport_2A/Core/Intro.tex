\quad During the second part of my gap year, I interned at Allianz Trade within the Economic Research department of the Paris team. 
Following my internship at AEW I was interested in joining a research department where I could be involved in publication work on a wide range of economic subjects. 
The Economic Research department is split between Paris (Allianz Trade) and Munich (Allianz SE) and produces weekly publications on high-profile topics. 
It ranges from reaction to events, analysis of certain data releases, updates on the teams’ view ahead of Central Banks meetings etc\dots
These publications are usually composed of 3 short reactionary or forward-looking \textit{‘stories’} (1 page long) and one longer \textit{‘feature’} (2/3 pages long). 
In addition to the weekly release the team also produces longer-term pieces. One or two a month are usually published. 
In the six months I spent within the team I had the chance to work on many different releases, including the annual Trade Survey report with my tutor, and I decided to focus on two specific ones. 

Firstly, I chose to present the work I did with Mr. Maxime Darmet that led to a feature tackling the issue of accrued savings since the pandemic in advanced economies (namely France, Germany, Spain, Italy, the UK, and the US). 
As the Covid crisis was disrupting most sectors of every economy, macroeconomic variables broke from their pre-pandemic trend. 
Incentives to consume were heavily curbed and governments launched support programs to assist their economy, especially with social transfers addressed to households. 
It is this disruption introduced in the data that we leveraged to compute the excess in disposable income relative to the shortfall in consumption. 
From the excess savings stock estimate for the whole economy, we also derived a per household stock in late 2022 that highlights which strata of the population potentially benefited the most from this period. 
We also gave a shot at estimating the form in which these stocks of excess savings were held in in the different countries of our study. 
Our work ended up being part of a publication\cite{az_savings}, but we also kept it as a monitoring file that we would update monthly during my time in the team. 
It is does not replace other models that economists have for the countries they monitor but rather comes as a secondary source with a different approach that could easily be updated by interns (should the methodology be changed to accomodate future changes).

Secondly, the larger piece of my report, relates to the work I did with Mrs. Roberta Fortes PhD and Mr. Darmet. 
It was in the team's agenda to publish a report on the European labour markets\cite{az_labour} at the end of the first semester of 2023, and they were two of the five economists involved in the process. 
I then had the chance to work on the wage part of the study where we assess wage development and inflationary pressures in the context of questioning a potential risk of wage-price loop. 
We evaluate to what extent macroeconomic variables such as labour market slack, inflation, and productivity growth account for wage dynamics, which consistent with economic theory should concur with demand-side drivers of inflation. 
We also evaluate different proxies for labour market slack since supply-side and demand-side indicators seem to have been diverging recently and imply different degrees of tightness for the labour market. 
In addition, we broaden our study by trying-out several slack measures as the unemployment rate alone may not adequately capture all constraints in the labour market leading to wage inflation. 
We assess recent drivers of wage developments using- from the estimated models- its variable decomposition which we also build on to produce forecast and an outlook for the different markets. 
Lastly, we provide a labour productivity breakdown of different European countries that highlights, as of late 2022, lasting structural differences after the Covid-19 pandemic.
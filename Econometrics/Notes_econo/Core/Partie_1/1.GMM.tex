%GMM
$\circlearrowleft$ \href[page=22]{file:///Users/prld/Desktop/git_proj/NoTeX/NoTeX/Econometrics/2A/Cours/chapitre1.pdf}{\textit{link-gmm}}\\
\noindent ($U_{i}=(Y_{i},X_{i},Z_{i})$) l'ensemble des données sur i\\
On veut estimer $\theta_{0}\in \mathbf{R^{K}}$ en utilisant : \blue{$E[g(U,\theta_{0})]=0$} \quad $g\mapsto \mathbf{R}^{L}, \; L\geqslant K$\par
Estimtateur GMM pas unique dans le cas $L>K$ car dépend de la matrice de pondération $\widehat{W}_{n}$ choisie.\par
Hypothèse d'identification $\Rightarrow$ GMM convergent et asymptotiquement normal.\par
Choix théorique $W_{0} = H^{-1} = V(g(U,\theta_{0}))^{-1} = E[g(U,\theta_{0}).g(U,\theta_{0})^{T}]^{-1}$ est \blue{optimal} : inconnu car fonction de $\theta_{0} \mapsto  W_{n} = \widehat{W}_{0}$\par
GMM est optimal $\mapsto$ on sous-entend qu'on choisit $W_{n} \cv W_{0} = H^{-1}$ \textit{matrice de pondération optimale}. \blue{Asymptotiquement efficace/optimal}

\bigbreak
\begin{boxH}
    \underline{\textbf{Cas exogène :}} \; $g(U,\beta) = X(Y-X'.\beta) \;,  \quad E[g(U,\beta_{0})]=0 \; \; \; L=K$\\
    Tous les choix de $\widehat{W}_{n}$ \textit{matrice de pondération} (symétrique, définie potentiellement aléatoire) conduisent au même estimateur = MCO
\end{boxH}
\bigbreak
\begin{boxH}
    \underline{\textbf{Cas instrumental :}} \; $g(U,\beta) = Z(Y-X'.\beta) \;,  \quad E[g(U,\beta_{0})]=0 \; \; \; Z\in \mathbf{R}^{L} \quad \textbf{L>K}$\\
    Différents $W_{n}$ $\longleftrightarrow$ différents estimateurs.
    \begin{itemize}
        \item [\textbf{-}] \textbf{Cas Homoscédastique : } l'estimateur GMM coïncide avec l'estimateur des 2MC (2SLS) $\Rightarrow$ \redline{2SLS est optimal dans le cas homoscédastique}
        \item [\textbf{-}] Sinon présence d'\textbf{hétéroscédastique} dans les résidus : GMM donne un nouvel estimateur $\neq$2SLS qui lui est bien optimal.
    \end{itemize}
\end{boxH}


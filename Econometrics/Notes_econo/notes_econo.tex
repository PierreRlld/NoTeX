\documentclass[12pt, french]{article}
\usepackage{times}
\usepackage[utf8]{inputenc}
\usepackage[T1]{fontenc}
\usepackage{lmodern,textcomp}
\usepackage[numbers,sort&compress]{natbib}
\usepackage[french]{babel}
\usepackage[a4paper,left=2cm,right=2cm,top=2cm,bottom=2cm]{geometry}
\usepackage{subfig}
\usepackage{hyperref}
\usepackage{graphicx}
\usepackage{fancyhdr}
\usepackage[dvipsnames]{xcolor}
\usepackage{caption}
\usepackage{float}
\usepackage{enumitem}
\usepackage{amsmath, amssymb}
\usepackage{amsfonts}
\usepackage{calc}
\usepackage{pdfpages}
\usepackage{bbold}
\usepackage{ulem}
\usepackage[many]{tcolorbox}


\usepackage{setspace}
\onehalfspacing

\definecolor{light-gray}{gray}{0.95}
\newcommand{\source}[1]{\caption*{Source: {#1}} }
\renewcommand{\baselinestretch}{1.4}

%math 
\newcommand{\indep}{\perp \!\!\! \perp}

%BOXES
\tcbset{
    sharp corners,
    colback = white,
    before skip = 0.2cm,    % add extra space before the box
    after skip = 0.5cm      % add extra space after the box
}    
\definecolor{main}{HTML}{e7e7e7}    
\definecolor{sub}{HTML}{F4F4F4}
\definecolor{sub2}{HTML}{f3f3f3}     
\newtcolorbox{boxH}{
    colback = sub, 
    colframe = main, 
    boxrule = 0pt, 
    leftrule = 6pt % left rule weight
}

%CUSTOM
\newcommand{\redline}[1]{\color{red}\uline{\textcolor{black}{#1}}\color{black}\:}
\newcommand{\ghl}[1]{\colorbox{sub2}{#1}\:}
\newcommand{\blue}[1]{\textcolor{blue}{#1}}
\newcommand{\red}[1]{\textcolor{BrickRed}{#1}}
\newcommand{\green}[1]{\textcolor{ForestGreen}{#1}}

%===INFO PDF===
\hypersetup{							% Information sur le document
pdfauthor = {Rouillard Pierre},			% Auteurs
pdftitle = {notes_econo},		            % Sujet
pdfstartview={FitH}}					% ajuste la page à la largeur de l'écran
%==============



%===START===
\begin{document}

% Page de garde
\newcommand{\HRule}{\rule{\linewidth}{0.5mm}}

\begin{titlepage}
\begin{center}

% Upper part of the page. The '~' is needed because only works if a paragraph has started.
\includegraphics[width=0.45\textwidth]{Core/logo.png}~\\[1cm]

%\textsc{\LARGE ENSAE Paris}\\[1.5cm]

\textsc{\Large }\\[0.5cm]

% Title
\begin{center}

\HRule \\[1cm]
{\huge \bfseries{Fiches d'économétrie 2}}\\[0.5cm]
\HRule \\[2.5cm]

\end{center}


% Author and supervisor
\begin{minipage}{0.4\textwidth}
\begin{flushleft} \large
\emph{Auteur:}\\
Pierre \textsc{Rouillard}\\


\end{flushleft}
\end{minipage}
\begin{minipage}{0.4\textwidth}
\begin{flushright} \large
%\emph{Référent:} \\
%Pierre-Antoine \textsc{Robert}
\end{flushright}
\end{minipage}

\vfill

% Bottom of the page
{\large \today}

\end{center}
\end{titlepage}
\newpage

% toc
\pagestyle{fancy}
\fancyhead[L]{}
\fancyhead[R]{}
\tableofcontents
\break

%page style
\pagestyle{fancy}
\fancyhead[L]{\leftmark}
\fancyhead[R]{}

% 1
\section{Points à savoir}
% 1.1
\subsection{Interprétation du paramètre causal à estimer}
Selon le modèle considéré il est possible ou non d'avoir une interprétation quantitative directe et/ou qualitative du paramètre causal à estimer $\beta_{0}$.\par
Définition de l'effet marginal de $X_{k}$ sur $Y$ : $\frac{\partial E[Y|X=x]}{\partial x_{k}}$
\bigbreak

%Modèle linéaire
\noindent $\hookrightarrow$ \underline{\textbf{Modèle linéaire :}}\\
Comme toujours par la suite on considère l'analyse \textit{toutes choses égales par ailleurs, sur la population considérée...}\par
\ghl{\textcolor{ForestGreen}{Interprétation} :} la variable d'intérêt est $Y$ et en l'absence de puissance ou d'interactions on peut interpréter quantitativement $\beta_{0}$ sur la variable d'intérêt. Le paramètre d'intérêt est l'effet marginal de $X_{k}$ sur $Y$ qui vaut bien $\beta_{0k}$ lorsque $X_{k}$ apparaît simplement dans le modèle. C'est justement pour cela qu'on peut bien interpréter directement quantitativement les coefficients de $\beta_{0}$ !
\bigbreak

%Modèle binaire
\noindent $\hookrightarrow$ \underline{\textbf{Modèle binaire :}} \par $E[Y|X]=P(Y=1|X)=F(X'\beta_{0})$ \par
$E[Y|X]=F(X'\beta_{0}) \; \Longleftrightarrow \; \; \; Y=\mathbb{1}(Y^{*} \geq s) \; : \; \; Y^{*}=X'\beta_{0} + \varepsilon \; \; \; \varepsilon \indep X $ \par
\ghl{Interprétation \textcolor{BrickRed}{quantitative directe} :} la variable d'intérêt est $Y$ et $Y^{*}$ n'est qu'une variable latente qui n'a pas forcément de sens quantitatif précis. Le paramètre d'intérêt est l'effet marginal de $X_{k}$ sur la variable d'intérêt, ici $Y$. Les coefficients de $\beta_{0}$ concernant $Y^{*}$ on ne peut donc pas directement interpréter quantitativement ces derniers sur la variable d'intérêt $Y$. De plus, l'effet marginal de $X_{k}$ sur $Y$ est différent de $\beta_{0k}$ : c'est pour cela qu'on ne peut avoir d'interprétation quantitative des coefficients de $\beta_{0}$ ! \par
\ghl{Interprétation \textcolor{ForestGreen}{qualitative} :} en revanche le signe de l'effet marginal de $X_{k}$ sur $Y$, i.e. effet positif ou négatif sur $P(Y=1|X)$, est donné par le signe de $\beta_{0k}$. \par
On peut en revanche comparer quantitativement le ratio des effets marginaux des variables i et j qui vaut $\widehat{\beta_{i}}/\widehat{\beta_{j}}$. \textit{L'effet sur la proba d'être ... de la variable i est <quantitativement> ... que l'effet de la variable j $\Longleftrightarrow$ regarder le rapport $\widehat{\beta_{i}}/\widehat{\beta_{j}}$ }.

\bigbreak
\noindent $\hookrightarrow$ \underline{\textbf{Modèle de censure / Tobit1 :}} \\
\textcolor{blue}{1 seul mécanisme détermine la valeur de $Y$ et si on observe la variable d'intérêt ou non. } Deux cas sont à distinguer :\par
\bigbreak
$\Rightarrow$ \textbf{\textcolor{ForestGreen}{Données censurées}} : la variable d'intérêt est $Y^{*}$ qui peut ne pas être observée au dessous d'un seuil causant un problème de censure. Le paramètre d'intérêt est l'effet marginal de $X_{k}$ sur la variable d'intérêt $Y^{*}$, qui vaut bien $\beta_{0k}$ lorsque $X_{k}$ apparaît simplement dans le modèle linéaire de $Y^{*}$. Ainsi, la variable $Y^{*}$ ayant un sens quantitatif et malgré la censure liée aux problèmes d'observation on peut bien interpréter quantitativement $\beta_{0}$ sur la variable d'intérêt.
\bigbreak
$\Rightarrow$ \textbf{\textcolor{BrickRed}{Solution en coin}} : la variable d'intérêt est bien $Y$ alors que la variable $Y^{*}$ est une variable latente potentiellement dépourvue de sens quantitatif. Typiquement un pb d'optimisation du consommateur où $Y^{*}$ mesure l'utilité optimale (en nombre de biens) de consommation d'un bien donné : donc potentiellement négatif. Et Y représente le nombre d'unités effectivement consommées. Les coefficients de $\beta_{0}$ concernant $Y^{*}$ qui n'as pas de sens quantitatif précis : on ne peut pas interpréter quantitativement les coefficients de $\beta_{0}$ sur la variable d'intérêt $Y$. Les paramètres d'intérêt sont les effets marginaux : le total $\frac{\partial E[Y|X=x]}{\partial x_{k}}$ (marge extensive et intensive) et $\frac{\partial E[Y|Y>0,X=x]}{\partial x_{k}}$ (marge intensive seulement). Ces paramètres sont tous les deux différents de $\beta_{0k}$ ce qui explique le manque d'interprétation quantitative des coefficients de $\beta_{0}$.
\bigbreak
\noindent $\hookrightarrow$ \underline{\textbf{Modèle de sélection / Tobit2 :}} \\
\textcolor{blue}{Ici on a bien deux processus différents : un qui détermine $Y^{*}$ \textbf{et un autre} qui détermine si on observe cette valeur ou non i.e. modèle sur D.}\par 
\ghl{Interprétation \textcolor{ForestGreen}{quantitative directe} :} il y a un problème d'observation des données, on observe $Y=D.Y^{*}$ mais la variable d'intérêt est bien $Y^{*}$ (\ghl{variable potentielle qui existe pour tous} \ghl{les \textit{individus}}). Par conséquent $Y^{*}$ suivant un modèle linéaire, les paramètres d'intérêts sont les effets marginaux des variables explicatives sur la variable d'intérêt $Y^{*}$ et les coefficients de $\beta_{0}$ sont toujours interprétables quantitativement.

\newpage
% 1.2
\subsection{Représentations linéaires \& OLS}
% Projection linéaire
\noindent $\hookrightarrow$ \underline{\textbf{Représentation non causale - Projection linéaire :}}\par
Sous conditions de moments \footnote{$E[Y^{2}]<+\infty$, $E[\lVert X \rVert^{2}]<+\infty$ et $E[XX']$ inversible = de rang plein. En particulier: \textbf{any level of correlation between covariates except perfect colinearity} : composantes de X linéairement indépendantes mais \textit{n'exclut pas} qu'elles soients corrélées. Si le modèle a une constante et une variable catégorielle il faut exclure une des modalités.}, on a toujours par construction/définition de la représentation linéaire théorique (=projection linéaire orthogonale) orthogonalité des \textit{résidus} de cette représentation non causale avec les régresseurs = pas une hypothèse mais une conséquence.\par
\begin{boxH}
    $Y = X'.\widetilde{\beta} + \widetilde{\varepsilon}$ , $E[X\widetilde{\varepsilon}]=0$ toujours définissable sous conditions de moments.\par
    \begin{itemize}
        \item[\textbf{-}] $\widehat{\beta_{OLS}}$ \redline{estime toujours} $\widetilde{\beta}$ : $\widehat{\beta_{OLS}} \underset{n \to +\infty}{\longrightarrow} \widetilde{\beta}$\par
        \item[\textbf{-}] $X'.\widetilde{\beta}$ meilleure prédiction linéaire de Y par X : $\widetilde{\beta}$ solution MSE.
    \end{itemize}
\end{boxH}

\begin{equation*} 
    \widehat{\beta_{OLS}} \in \underset{\beta}{\mathrm{argmin}} \sum_{i = 1}^{n}{(Y_{i} - X_{i}'.\beta)^{2}} \quad \longleftrightarrow \quad \widetilde{\beta} \in \underset{\beta}{\mathrm{argmin}} \; E[(Y-X'.\beta)^{2}]
\end{equation*}

\bigbreak
% Représentation causale
\noindent $\hookrightarrow$ \underline{\textbf{Représentation causale :}}\par
La représentation causale fait intervenir \ghl{le paramètre causal $\beta_{0}$ qu'on cherche à estimer} : dans cette représentation le \textit{terme d'erreur} n'est pas automatiquement orthogonal au régresseur.\par
\begin{boxH}
    $Y = X'.\beta_{0} + \varepsilon$ , $E[X\varepsilon]\overset{\textbf{\red{?}}}{=}0$\par
    \begin{itemize}
        \item[\textbf{-}] $\beta_{0}$ paramètre causal à estimer.\par
        \item[\textbf{-}] $\varepsilon$ résidu : agrège les facteurs inobservés qui affectent $Y$.
    \end{itemize}
\end{boxH}
Le \ghl{terme d'erreur $\varepsilon$ capte l'hétérogénéité inobservée}, i.e capte les déterminants inobservés qui affectent la variable d'intérêt $Y$ : deux individus avec les mêmes variables explicatives auront néanmoins la plupart du temps des variables expliquées différentes.\par
Avoir orthogonalité ($\rightarrow$ indépendance) entre régresseurs et terme d'erreur est une hypothèse ! C'est \textbf{l'hypothèse d'exogénéité}.

\bigbreak
%Lien
\noindent $\hookrightarrow$ \underline{\textbf{Lien :}}\par
\blue{Sans l'hypothèse d'exogénéité pour la représentation causale, les deux représentations diffèrent} et l'estimateur OLS ne permet pas d'identifier le paramètre causal d'intérêt.\par
En revanche \ghl{avec hypothèse d'exogénéité les deux représentations coïncident} et $\widetilde{\beta} = \beta_{0}$ : $\widehat{\beta_{OLS}}$ qui estime toujours $\widetilde{\beta}$ est donc un estimateur consistant de $\beta_{0}$.\par
En dehors des expériences contrôlées les variables explicatives peuvent parfois être corrélées aux facteurs inobersables et pb d'endogénéité $E[X\varepsilon]\neq 0$.
\bigbreak


\subsection{Résumé modèles}
%\input{Core/Partie_0/0.ii}

\newpage

% 2
\section{OLS - MCO}
%\input{Core/Partie_1/Partie_1}
% 2.1
\subsection{Rappels MCO}
%Selon le modèle considéré il est possible ou non d'avoir une interprétation quantitative directe et/ou qualitative du paramètre causal à estimer $\beta_{0}$.\par
Définition de l'effet marginal de $X_{k}$ sur $Y$ : $\frac{\partial E[Y|X=x]}{\partial x_{k}}$
\bigbreak

%Modèle linéaire
\noindent $\hookrightarrow$ \underline{\textbf{Modèle linéaire :}}\\
Comme toujours par la suite on considère l'analyse \textit{toutes choses égales par ailleurs, sur la population considérée...}\par
\ghl{\textcolor{ForestGreen}{Interprétation} :} la variable d'intérêt est $Y$ et en l'absence de puissance ou d'interactions on peut interpréter quantitativement $\beta_{0}$ sur la variable d'intérêt. Le paramètre d'intérêt est l'effet marginal de $X_{k}$ sur $Y$ qui vaut bien $\beta_{0k}$ lorsque $X_{k}$ apparaît simplement dans le modèle. C'est justement pour cela qu'on peut bien interpréter directement quantitativement les coefficients de $\beta_{0}$ !
\bigbreak

%Modèle binaire
\noindent $\hookrightarrow$ \underline{\textbf{Modèle binaire :}} \par $E[Y|X]=P(Y=1|X)=F(X'\beta_{0})$ \par
$E[Y|X]=F(X'\beta_{0}) \; \Longleftrightarrow \; \; \; Y=\mathbb{1}(Y^{*} \geq s) \; : \; \; Y^{*}=X'\beta_{0} + \varepsilon \; \; \; \varepsilon \indep X $ \par
\ghl{Interprétation \textcolor{BrickRed}{quantitative directe} :} la variable d'intérêt est $Y$ et $Y^{*}$ n'est qu'une variable latente qui n'a pas forcément de sens quantitatif précis. Le paramètre d'intérêt est l'effet marginal de $X_{k}$ sur la variable d'intérêt, ici $Y$. Les coefficients de $\beta_{0}$ concernant $Y^{*}$ on ne peut donc pas directement interpréter quantitativement ces derniers sur la variable d'intérêt $Y$. De plus, l'effet marginal de $X_{k}$ sur $Y$ est différent de $\beta_{0k}$ : c'est pour cela qu'on ne peut avoir d'interprétation quantitative des coefficients de $\beta_{0}$ ! \par
\ghl{Interprétation \textcolor{ForestGreen}{qualitative} :} en revanche le signe de l'effet marginal de $X_{k}$ sur $Y$, i.e. effet positif ou négatif sur $P(Y=1|X)$, est donné par le signe de $\beta_{0k}$. \par
On peut en revanche comparer quantitativement le ratio des effets marginaux des variables i et j qui vaut $\widehat{\beta_{i}}/\widehat{\beta_{j}}$. \textit{L'effet sur la proba d'être ... de la variable i est <quantitativement> ... que l'effet de la variable j $\Longleftrightarrow$ regarder le rapport $\widehat{\beta_{i}}/\widehat{\beta_{j}}$ }.

\bigbreak
\noindent $\hookrightarrow$ \underline{\textbf{Modèle de censure / Tobit1 :}} \\
\textcolor{blue}{1 seul mécanisme détermine la valeur de $Y$ et si on observe la variable d'intérêt ou non. } Deux cas sont à distinguer :\par
\bigbreak
$\Rightarrow$ \textbf{\textcolor{ForestGreen}{Données censurées}} : la variable d'intérêt est $Y^{*}$ qui peut ne pas être observée au dessous d'un seuil causant un problème de censure. Le paramètre d'intérêt est l'effet marginal de $X_{k}$ sur la variable d'intérêt $Y^{*}$, qui vaut bien $\beta_{0k}$ lorsque $X_{k}$ apparaît simplement dans le modèle linéaire de $Y^{*}$. Ainsi, la variable $Y^{*}$ ayant un sens quantitatif et malgré la censure liée aux problèmes d'observation on peut bien interpréter quantitativement $\beta_{0}$ sur la variable d'intérêt.
\bigbreak
$\Rightarrow$ \textbf{\textcolor{BrickRed}{Solution en coin}} : la variable d'intérêt est bien $Y$ alors que la variable $Y^{*}$ est une variable latente potentiellement dépourvue de sens quantitatif. Typiquement un pb d'optimisation du consommateur où $Y^{*}$ mesure l'utilité optimale (en nombre de biens) de consommation d'un bien donné : donc potentiellement négatif. Et Y représente le nombre d'unités effectivement consommées. Les coefficients de $\beta_{0}$ concernant $Y^{*}$ qui n'as pas de sens quantitatif précis : on ne peut pas interpréter quantitativement les coefficients de $\beta_{0}$ sur la variable d'intérêt $Y$. Les paramètres d'intérêt sont les effets marginaux : le total $\frac{\partial E[Y|X=x]}{\partial x_{k}}$ (marge extensive et intensive) et $\frac{\partial E[Y|Y>0,X=x]}{\partial x_{k}}$ (marge intensive seulement). Ces paramètres sont tous les deux différents de $\beta_{0k}$ ce qui explique le manque d'interprétation quantitative des coefficients de $\beta_{0}$.
\bigbreak
\noindent $\hookrightarrow$ \underline{\textbf{Modèle de sélection / Tobit2 :}} \\
\textcolor{blue}{Ici on a bien deux processus différents : un qui détermine $Y^{*}$ \textbf{et un autre} qui détermine si on observe cette valeur ou non i.e. modèle sur D.}\par 
\ghl{Interprétation \textcolor{ForestGreen}{quantitative directe} :} il y a un problème d'observation des données, on observe $Y=D.Y^{*}$ mais la variable d'intérêt est bien $Y^{*}$ (\ghl{variable potentielle qui existe pour tous} \ghl{les \textit{individus}}). Par conséquent $Y^{*}$ suivant un modèle linéaire, les paramètres d'intérêts sont les effets marginaux des variables explicatives sur la variable d'intérêt $Y^{*}$ et les coefficients de $\beta_{0}$ sont toujours interprétables quantitativement.

\newpage





\end{document}
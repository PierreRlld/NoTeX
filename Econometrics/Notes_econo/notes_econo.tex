\documentclass[12pt, french]{article}
\usepackage{times}
\usepackage[utf8]{inputenc}
\usepackage[T1]{fontenc}
\usepackage{lmodern,textcomp}
\usepackage[numbers,sort&compress]{natbib}
\usepackage[french]{babel}
\usepackage[a4paper,left=2cm,right=2cm,top=2cm,bottom=2cm]{geometry}
\usepackage{subfig}
\usepackage{hyperref}
\usepackage{graphicx}
\usepackage{fancyhdr}
\usepackage[dvipsnames]{xcolor}
\usepackage{caption}
\usepackage{float}
\usepackage{enumitem}
\usepackage{amsmath, amssymb}
\usepackage{amsfonts}
\usepackage{calc}
\usepackage{pdfpages}
\usepackage{bbold}
\usepackage{ulem}
\usepackage[many]{tcolorbox}


\usepackage{setspace}
\onehalfspacing

\definecolor{light-gray}{gray}{0.95}
\newcommand{\source}[1]{\caption*{Source: {#1}} }
\renewcommand{\baselinestretch}{1.4}

%math 
\newcommand{\indep}{\perp \!\!\! \perp}
\newcommand{\innerproduct}[2]{\langle #1, #2 \rangle}
\newcommand{\cv}{\underset{n \to +\infty}{\longrightarrow}}

%BOXES
\tcbset{
    sharp corners,
    colback = white,
    before skip = 0.2cm,    % add extra space before the box
    after skip = 0.5cm      % add extra space after the box
}    
\definecolor{main}{HTML}{e7e7e7}    
\definecolor{sub}{HTML}{F4F4F4}
\definecolor{sub2}{HTML}{f3f3f3}     
\newtcolorbox{boxH}{
    colback = sub, 
    colframe = main, 
    boxrule = 0pt, 
    leftrule = 6pt % left rule weight
}

%CUSTOM
\newcommand{\redline}[1]{\color{red}\uline{\textcolor{black}{#1}}\color{black}\:}
\newcommand{\ghl}[1]{\colorbox{sub2}{#1}\:}
\newcommand{\blue}[1]{\textcolor{blue}{#1}}
\newcommand{\red}[1]{\textcolor{BrickRed}{#1}}
\newcommand{\green}[1]{\textcolor{ForestGreen}{#1}}

\hypersetup{pdfauthor = {Rouillard Pierre}, pdftitle = {notes-econo}, pdfstartview={FitH}}	

%===START===
\begin{document}

% Page de garde
\newcommand{\HRule}{\rule{\linewidth}{0.5mm}}

\begin{titlepage}
\begin{center}

\begin{minipage}{1\textwidth}
\Large{\textbf{ROUILLARD Pierre}} 
\hfill
\large{\textbf{ENSAE 2\textsuperscript{ère} année}}\par
\vspace{0cm}
\hfill \normalsize{\textit{Stage long}}\par
\hfill \normalsize{\textit{Année scolaire 2022-2023}}  
\end{minipage}

\vspace{8cm}

% Title
\begin{center}

\HRule \\[1cm]
{\huge \bfseries{Stage d'Application}}\\[0.5cm]
\HRule \\[2.5cm]

\end{center}

\vfill

% Bottom
\begin{minipage}{1\textwidth}
\begin{flushleft}
\large{\textbf{\textsc{Allianz Trade}}} \hfill \small{Maître de stage :} \large{\textbf{\textsc{Françoise HUANG}}} \\ 
\small{Paris, France} \hfill \small{3 Janvier - 26 Juin 2023}
\end{flushleft}
\end{minipage}

\end{center}
\end{titlepage}
\newpage

% toc
\pagestyle{fancy}
\fancyhead[L]{}
\fancyhead[R]{}
\tableofcontents
\break

%page style
\pagestyle{fancy}
\fancyhead[L]{\leftmark}
\fancyhead[R]{}


\section{Econo2}

\subsection{Interprétation du paramètre causal à estimer}
%###
Selon le modèle considéré il est possible ou non d'avoir une interprétation quantitative directe et/ou qualitative du paramètre causal à estimer $\beta_{0}$.\par
Définition de l'effet marginal de $X_{k}$ sur $Y$ : $\frac{\partial E[Y|X=x]}{\partial x_{k}}$
\bigbreak

%Modèle linéaire
\noindent $\hookrightarrow$ \underline{\textbf{Modèle linéaire :}}\\
Comme toujours par la suite on considère l'analyse \textit{toutes choses égales par ailleurs, sur la population considérée...}\par
\ghl{\textcolor{ForestGreen}{Interprétation} :} la variable d'intérêt est $Y$ et en l'absence de puissance ou d'interactions on peut interpréter quantitativement $\beta_{0}$ sur la variable d'intérêt. Le paramètre d'intérêt est l'effet marginal de $X_{k}$ sur $Y$ qui vaut bien $\beta_{0k}$ lorsque $X_{k}$ apparaît simplement dans le modèle. C'est justement pour cela qu'on peut bien interpréter directement quantitativement les coefficients de $\beta_{0}$ !
\bigbreak

%Modèle binaire
\noindent $\hookrightarrow$ \underline{\textbf{Modèle binaire :}} \par $E[Y|X]=P(Y=1|X)=F(X'\beta_{0})$ \par
$E[Y|X]=F(X'\beta_{0}) \; \Longleftrightarrow \; \; \; Y=\mathbb{1}(Y^{*} \geq s) \; : \; \; Y^{*}=X'\beta_{0} + \varepsilon \; \; \; \varepsilon \indep X $ \par
\ghl{Interprétation \textcolor{BrickRed}{quantitative directe} :} la variable d'intérêt est $Y$ et $Y^{*}$ n'est qu'une variable latente qui n'a pas forcément de sens quantitatif précis. Le paramètre d'intérêt est l'effet marginal de $X_{k}$ sur la variable d'intérêt, ici $Y$. Les coefficients de $\beta_{0}$ concernant $Y^{*}$ on ne peut donc pas directement interpréter quantitativement ces derniers sur la variable d'intérêt $Y$. De plus, l'effet marginal de $X_{k}$ sur $Y$ est différent de $\beta_{0k}$ : c'est pour cela qu'on ne peut avoir d'interprétation quantitative des coefficients de $\beta_{0}$ ! \par
\ghl{Interprétation \textcolor{ForestGreen}{qualitative} :} en revanche le signe de l'effet marginal de $X_{k}$ sur $Y$, i.e. effet positif ou négatif sur $P(Y=1|X)$, est donné par le signe de $\beta_{0k}$. \par
On peut en revanche comparer quantitativement le ratio des effets marginaux des variables i et j qui vaut $\widehat{\beta_{i}}/\widehat{\beta_{j}}$. \textit{L'effet sur la proba d'être ... de la variable i est <quantitativement> ... que l'effet de la variable j $\Longleftrightarrow$ regarder le rapport $\widehat{\beta_{i}}/\widehat{\beta_{j}}$ }.

\bigbreak
\noindent $\hookrightarrow$ \underline{\textbf{Modèle de censure / Tobit1 :}} \\
\textcolor{blue}{1 seul mécanisme détermine la valeur de $Y$ et si on observe la variable d'intérêt ou non. } Deux cas sont à distinguer :\par
\bigbreak
$\Rightarrow$ \textbf{\textcolor{ForestGreen}{Données censurées}} : la variable d'intérêt est $Y^{*}$ qui peut ne pas être observée au dessous d'un seuil causant un problème de censure. Le paramètre d'intérêt est l'effet marginal de $X_{k}$ sur la variable d'intérêt $Y^{*}$, qui vaut bien $\beta_{0k}$ lorsque $X_{k}$ apparaît simplement dans le modèle linéaire de $Y^{*}$. Ainsi, la variable $Y^{*}$ ayant un sens quantitatif et malgré la censure liée aux problèmes d'observation on peut bien interpréter quantitativement $\beta_{0}$ sur la variable d'intérêt.
\bigbreak
$\Rightarrow$ \textbf{\textcolor{BrickRed}{Solution en coin}} : la variable d'intérêt est bien $Y$ alors que la variable $Y^{*}$ est une variable latente potentiellement dépourvue de sens quantitatif. Typiquement un pb d'optimisation du consommateur où $Y^{*}$ mesure l'utilité optimale (en nombre de biens) de consommation d'un bien donné : donc potentiellement négatif. Et Y représente le nombre d'unités effectivement consommées. Les coefficients de $\beta_{0}$ concernant $Y^{*}$ qui n'as pas de sens quantitatif précis : on ne peut pas interpréter quantitativement les coefficients de $\beta_{0}$ sur la variable d'intérêt $Y$. Les paramètres d'intérêt sont les effets marginaux : le total $\frac{\partial E[Y|X=x]}{\partial x_{k}}$ (marge extensive et intensive) et $\frac{\partial E[Y|Y>0,X=x]}{\partial x_{k}}$ (marge intensive seulement). Ces paramètres sont tous les deux différents de $\beta_{0k}$ ce qui explique le manque d'interprétation quantitative des coefficients de $\beta_{0}$.
\bigbreak
\noindent $\hookrightarrow$ \underline{\textbf{Modèle de sélection / Tobit2 :}} \\
\textcolor{blue}{Ici on a bien deux processus différents : un qui détermine $Y^{*}$ \textbf{et un autre} qui détermine si on observe cette valeur ou non i.e. modèle sur D.}\par 
\ghl{Interprétation \textcolor{ForestGreen}{quantitative directe} :} il y a un problème d'observation des données, on observe $Y=D.Y^{*}$ mais la variable d'intérêt est bien $Y^{*}$ (\ghl{variable potentielle qui existe pour tous} \ghl{les \textit{individus}}). Par conséquent $Y^{*}$ suivant un modèle linéaire, les paramètres d'intérêts sont les effets marginaux des variables explicatives sur la variable d'intérêt $Y^{*}$ et les coefficients de $\beta_{0}$ sont toujours interprétables quantitativement.

\newpage
\subsection{Représentations linéaires \& OLS}
% Projection linéaire
\noindent \textbf{}
\noindent On note $Y \in \mathbf{R}$ la variable d'intérêt/dépendante et $X \in \mathbf{R}^{\textbf{K}}$ le vecteur des variables explicatives.
\bigbreak
\noindent $\hookrightarrow$ \underline{\textbf{Représentation non causale - Projection linéaire :}}\par
Sous conditions de moments \footnote{$E[Y^{2}]<+\infty$, $E[\lVert X \rVert^{2}]<+\infty$ et $E[XX']$ inversible = de rang plein. En particulier: \textbf{any level of correlation between covariates except perfect colinearity} : composantes de X linéairement indépendantes mais \textit{n'exclut pas} qu'elles soients corrélées. Si le modèle a une constante et une variable catégorielle il faut exclure une des modalités.}, on a toujours par construction/définition de la représentation linéaire théorique (=projection linéaire orthogonale) orthogonalité des \textit{résidus} de cette représentation non causale avec les régresseurs = pas une hypothèse mais une conséquence.\par
\begin{boxH}
    $Y = X'.\widetilde{\beta} + \widetilde{\varepsilon}$ , $E[X\widetilde{\varepsilon}]=0$ toujours définissable sous conditions de moments.\par
    \begin{itemize}
        \item[\textbf{-}] $\widehat{\beta}_{OLS}$ \redline{estime toujours} $\widetilde{\beta}$ : $\widehat{\beta}_{OLS} \underset{n \to +\infty}{\longrightarrow} \widetilde{\beta}$\par
        \item[\textbf{-}] $X'.\widetilde{\beta}$ meilleure prédiction linéaire de Y par X : $\widetilde{\beta}$ solution MSE.
    \end{itemize}
\end{boxH}

\begin{equation*} 
    \widehat{\beta}_{OLS} \in \underset{\beta}{\mathrm{argmin}} \sum_{i = 1}^{n}{(Y_{i} - X_{i}'.\beta)^{2}} \quad \longleftrightarrow \quad \widetilde{\beta} \in \underset{\beta}{\mathrm{argmin}} \; E[(Y-X'.\beta)^{2}]
\end{equation*}

\bigbreak
% Représentation causale
\noindent $\hookrightarrow$ \underline{\textbf{Représentation causale :}}\par
La représentation causale fait intervenir \ghl{le paramètre causal $\beta_{0}$ qu'on cherche à estimer} : dans cette représentation le \textit{terme d'erreur} n'est pas automatiquement orthogonal au régresseur.\par
\begin{boxH}
    $Y = X'.\beta_{0} + \varepsilon$ , $E[X\varepsilon]\overset{\textbf{\red{?}}}{=}0$\par
    \begin{itemize}
        \item[\textbf{-}] $\beta_{0}$ paramètre causal à estimer.\par
        \item[\textbf{-}] $\varepsilon$ résidu : agrège les facteurs inobservés qui affectent $Y$.
    \end{itemize}
\end{boxH}
Le \ghl{terme d'erreur $\varepsilon$ capte l'hétérogénéité inobservée}, i.e capte les déterminants inobservés qui affectent la variable d'intérêt $Y$ : deux individus avec les mêmes variables explicatives auront néanmoins la plupart du temps des variables expliquées différentes.\par
\blue{Avoir orthogonalité ($\rightarrow$ indépendance) entre régresseurs et terme d'erreur est une hypothèse !} C'est \textbf{l'hypothèse d'exogénéité}.

\bigbreak
%Lien
\noindent $\hookrightarrow$ \underline{\textbf{Lien :}}\par
\blue{Sans l'hypothèse d'exogénéité pour la représentation causale, les deux représentations diffèrent} et l'estimateur OLS ne permet pas d'identifier le paramètre causal d'intérêt.\par
En revanche \ghl{avec hypothèse d'exogénéité les deux représentations coïncident} et $\widetilde{\beta} = \beta_{0}$ : $\widehat{\beta}_{OLS}$ qui estime toujours $\widetilde{\beta}$ est donc un estimateur consistant de $\beta_{0}$.\par
En dehors des expériences contrôlées les variables explicatives peuvent parfois être corrélées aux facteurs inobersables et pb d'endogénéité $E[X\varepsilon]\neq 0$.
\bigbreak

%Mémo OLS
\noindent $\hookrightarrow$ \underline{\textbf{Mémo OLS :}}\par
$\circlearrowleft$ \href[page=5]{file:///Users/prld/Desktop/git_proj/NoTeX/NoTeX/Econometrics/2A/Cours/chapitre1.pdf}{\textit{link-ols}}\\
Représentation causale
\begin{boxH}
    \textbf{Estimateur OLS - MCO}\par
    \quad $ Y = X'.\beta_{0} + \varepsilon $ , avec $E[\varepsilon] = 0$ 
    \begin{itemize}
        \item[\textbf{-}] Conditions de moments
        \item[\textbf{-}] $ E[X\varepsilon] = 0 $ hypothèse d'exogénéité
    \end{itemize}
$\Rightarrow \beta_{0} $ est identifiable $ \beta_{0} = E[XX']^{-1}E[XY]$ \par
$\Rightarrow $ L'estimateur $\widehat{\beta}_{OLS}$ est consistant, sans biais \& asymptotiquement normal
\end{boxH} 

On a:
\begin{equation*}
    \widehat{\beta}_{OLS} =  \left( \frac{1}{n}\sum_{i = 1}^{n}X_{i}X_{i}' \right)^{-1}\left( \frac{1}{n}\sum_{i = 1}^{n}X_{i}Y_{i} \right)
\end{equation*}
\begin{equation*}
    \sqrt{n}(\widehat{\beta}_{OLS} - \beta_{0}) \underset{n \to +\infty}{\longrightarrow} \mathcal{N}(O,\redline{E[XX']^{-1}E[\varepsilon^{2}XX']E[XX']^{-1}})
\end{equation*}
\vspace*{-0.75cm}

%Homoscédasticité
\noindent \textbf{Homoscédasticité}: hypothèse sur la variance des résidus. L'erreur standard des MCO diffèrent en fonction de l'hypothèse!
\begin{itemize}
    \item[\textbf{-}] Faible $E[\varepsilon^{2}XX'] = E[\varepsilon^{2}]E[XX']$
    \item[\textbf{-}] Forte $E[\varepsilon^{2}\vert X] = \sigma^{2} \leftrightarrow V[\varepsilon\vert X] = \sigma^{2}$ 
\end{itemize}
\noindent L'hypothèse d'homoscédasticité forte requiert que la variance des termes d'erreur soit la même pour chaque observation.
\blue{L'estimateur $\widehat{V}$ de la variance asymptotique de $\widehat{\beta}_{OLS}$ est robuste à l'hétéroscédasticité. L'estimateur standard $\widetilde{V}$ n'est convergent que sous l'hypothèse d'homoscédasticité.}
\begin{align*}
    \underline{Robust} \quad \widehat{V} &= \left(\frac{1}{n-k}\sum_{i = 1}^{n}X_{i}X_{i}'\right)^{-1} \left(\frac{1}{n-k}\sum_{i = 1}^{n}\hat{\varepsilon}^2_{i}X_{i}X_{i}'\right) \left(\frac{1}{n-k}\sum_{i = 1}^{n}X_{i}X_{i}'\right)^{-1} \\    
    \underline{Standard} \quad \widetilde{V} &= \left(\frac{1}{n-k}\sum_{i = 1}^{n}\hat{\varepsilon}^2_{i}\right)\left(\frac{1}{n-k}\sum_{i = 1}^{n}X_{i}X_{i}'\right)^{-1} \; \; \textrm{\ghl{sous hypothèse d'homoscédasticité}}
\end{align*}    


\newpage
\subsection{Modèle I.V - estimateur 2SLS/2MC}
$\circlearrowleft$ \href[page=9]{file:///Users/prld/Desktop/git_proj/NoTeX/NoTeX/Econometrics/2A/Cours/chapitre1.pdf}{\textit{link-2sls}}\\
% Variables instrumentales
\noindent On suppose $Y = X'.\beta_{0} + \varepsilon$ avec \red{problème d'\textbf{endogénéité} $E[X\varepsilon]\neq 0$} \quad 
\begin{boxH}
    \textbf{Variables instrumentales:} $X\in \mathbf{R^{K}}$, $Z\in \mathbf{R^{L}}$, avec $L\geqslant K$
    \begin{enumerate}
        \item Exogénéité: $E[Z\varepsilon]=0$
        \item Condition de rang $E[ZX']$ de rang $K$ $\leftrightarrow$ Pertinence $Cov(Z, X^{(i)})\neq 0 \quad \forall i$
    \end{enumerate}        
\end{boxH}

\noindent Condition de rang donne qu'il exite $\Gamma$ tq $\Gamma.E[ZX']$ inversible. \textit{Condition de rang} testable avec first step (\ghl{significativité d'un des coeff. des vrais instruments}) et \textit{exogénéité} pas testable.\\ 
Si $L=K$ alors $\beta_{0}$ est \textit{juste identifié}, si $L>K$ alors $\beta_{0}$ est \textit{suridentifié}.\\
\blue{Z inclut toutes les variables exogènes.} En pratique on ne régresse et remplace par le projection linéaire sur Z estimée que les variables endogènes.

%Estimateur 2SLS
\begin{boxH}
    \textbf{Estimateur 2MC - 2SLS}
    \begin{enumerate}
        \item Projection linéaire de X sur Z $\mapsto X^{*} = \Gamma Z \\ \textrm{où} \; \Gamma = E[XZ']E[ZZ']^{-1} = (\beta ^{\!^{\;OLS'}}_{(1)/Z},...,\beta ^{\!^{\;OLS'}}_{(K-1)/Z})^{T} = "\beta ^{\!^{\;OLS}}_{X/Z}" $
        \item Reg lin de Y sur $X^{*}$ $\mapsto \textrm{OLS estimé est } \widehat{\beta}_{2SLS}$
    \end{enumerate}
\end{boxH}

\begin{enumerate}
    \item[.] $Y = X'.\beta_{0} + \varepsilon$
    \item[.] $\beta_{0} = E[\Gamma ZX']^{-1}E[\Gamma ZY] = E[X^{*}X']^{-1}E[X^{*}Y] $
    \item[or] $\innerproduct{z}{x-p_{\mathbf{Z}}(x)}=0 \quad \forall z\in \mathbf{Z} \quad \Rightarrow \quad z=X^{*}\in Vect(Z) \quad E[X^{*}X^{T}]=E[X^{*}X^{*T}]$
    \item[.] \blue{$\beta_{0} = E[X^{*}X^{*'}]^{-1}E[X^{*}Y]$}
\end{enumerate}

\begin{boxH}
    Estimateur doubles moindres carrés: $\hat{\beta}_{2SLS} \cv \beta_{0}$
    \vspace*{-0.5cm}
    \begin{equation*}
        \hat{\beta}_{2SLS} = \left( \frac{1}{n}\sum_{i = 1}^{n}\hat{X}_{i}\hat{X}_{i}' \right)^{-1}\left( \frac{1}{n}\sum_{i = 1}^{n}\hat{X}_{i}Y_{i} \right) \quad \textrm{où} \;\; \hat{X}_{i} = \red{\hat{\Gamma}}Z_{i} \quad \red{\hat{\Gamma} = (\hat{\beta}^{\!^{\;OLS'}}_{(1)/Z},...,\hat{\beta}^{\!^{\;OLS'}}_{(K-1)/Z})^{T}}
        \vspace*{-0.25cm}
    \end{equation*}
    
    $\Rightarrow $ L'estimateur $\hat{\beta}_{2SLS}$ est convergent \& asymptotiquement normal \textbf{mais} pas necéssairement sans biais.
\end{boxH}
\vspace*{-1cm}
\begin{equation*}
    \sqrt{n}(\widehat{\beta}_{2SLS} - \beta_{0}) \cv \mathcal{N}(O,\textrm{VA($\widehat{\beta}_{2SLS}$)}=\redline{E[X^{*}X^{*'}]^{-1}E[\varepsilon^{2}X^{*}X^{*'}]E[X^{*}X^{*'}]^{-1}})
\end{equation*}
$\hookrightarrow \varepsilon$ estimé par $\widehat{\varepsilon}_{i} = Y_{i} - \boldsymbol{X'_{i}}.\widehat{\beta}_{2SLS} \;$ et \blue{$\widehat{\textrm{VA}}$($\widehat{\beta}_{2SLS}$) estimateur convergent robuste (car ne repose pas sur des hypothèses d'homoscédasticité) de la variance asymptotique}. Ce n'est pas l'estimateur qu'on obtient si l'on fait une régression de $Y$ sur $\widehat{X}$ \ghl{car $\widehat{\varepsilon}_{i} \neq Y_{i} - \widehat{X}_{i}'.\widehat{\beta}_{2SLS}$}
\subsection{Méthode des moments généralisés}
%GMM
$\circlearrowleft$ \href[page=22]{file:///Users/prld/Desktop/git_proj/NoTeX/NoTeX/Econometrics/2A/Cours/chapitre1.pdf}{\textit{link-gmm}}\\
\noindent ($U_{i}=(Y_{i},X_{i},Z_{i})$) l'ensemble des données sur i\\
On veut estimer $\theta_{0}\in \mathbf{R^{K}}$ en utilisant : \blue{$E[g(U,\theta_{0})]=0$} \quad $g\mapsto \mathbf{R}^{L}, \; L\geqslant K$\par
Estimtateur GMM pas unique dans le cas $L>K$ car dépend de la matrice de pondération $\widehat{W}_{n}$ choisie.\par
Hypothèse d'identification $\Rightarrow$ GMM convergent et asymptotiquement normal.\par
Choix théorique $W_{0} = H^{-1} = V(g(U,\theta_{0}))^{-1} = E[g(U,\theta_{0}).g(U,\theta_{0})^{T}]^{-1}$ est \blue{optimal} : inconnu car fonction de $\theta_{0} \mapsto  W_{n} = \widehat{W}_{0}$\par
GMM est optimal $\mapsto$ on sous-entend qu'on choisit $W_{n} \cv W_{0} = H^{-1}$ \textit{matrice de pondération optimale}. \blue{Asymptotiquement efficace/optimal}

\bigbreak
\begin{boxH}
    \underline{\textbf{Cas exogène :}} \; $g(U,\beta) = X(Y-X'.\beta) \;,  \quad E[g(U,\beta_{0})]=0 \; \; \; L=K$\\
    Tous les choix de $\widehat{W}_{n}$ \textit{matrice de pondération} (symétrique, définie potentiellement aléatoire) conduisent au même estimateur = MCO
\end{boxH}
\bigbreak
\begin{boxH}
    \underline{\textbf{Cas instrumental :}} \; $g(U,\beta) = Z(Y-X'.\beta) \;,  \quad E[g(U,\beta_{0})]=0 \; \; \; Z\in \mathbf{R}^{L} \quad \textbf{L>K}$\\
    Différents $W_{n}$ $\longleftrightarrow$ différents estimateurs.
    \begin{itemize}
        \item [\textbf{-}] \textbf{Cas Homoscédastique : } l'estimateur GMM coïncide avec l'estimateur des 2MC (2SLS) $\Rightarrow$ \redline{2SLS est optimal dans le cas homoscédastique}
        \item [\textbf{-}] Sinon présence d'\textbf{hétéroscédastique} dans les résidus : GMM donne un nouvel estimateur $\neq$2SLS qui lui est bien optimal.
    \end{itemize}
\end{boxH}



\newpage

\section{XXX}
%\input{Core/Partie_1/Partie_1}
\subsection{xxx}
%%###
Selon le modèle considéré il est possible ou non d'avoir une interprétation quantitative directe et/ou qualitative du paramètre causal à estimer $\beta_{0}$.\par
Définition de l'effet marginal de $X_{k}$ sur $Y$ : $\frac{\partial E[Y|X=x]}{\partial x_{k}}$
\bigbreak

%Modèle linéaire
\noindent $\hookrightarrow$ \underline{\textbf{Modèle linéaire :}}\\
Comme toujours par la suite on considère l'analyse \textit{toutes choses égales par ailleurs, sur la population considérée...}\par
\ghl{\textcolor{ForestGreen}{Interprétation} :} la variable d'intérêt est $Y$ et en l'absence de puissance ou d'interactions on peut interpréter quantitativement $\beta_{0}$ sur la variable d'intérêt. Le paramètre d'intérêt est l'effet marginal de $X_{k}$ sur $Y$ qui vaut bien $\beta_{0k}$ lorsque $X_{k}$ apparaît simplement dans le modèle. C'est justement pour cela qu'on peut bien interpréter directement quantitativement les coefficients de $\beta_{0}$ !
\bigbreak

%Modèle binaire
\noindent $\hookrightarrow$ \underline{\textbf{Modèle binaire :}} \par $E[Y|X]=P(Y=1|X)=F(X'\beta_{0})$ \par
$E[Y|X]=F(X'\beta_{0}) \; \Longleftrightarrow \; \; \; Y=\mathbb{1}(Y^{*} \geq s) \; : \; \; Y^{*}=X'\beta_{0} + \varepsilon \; \; \; \varepsilon \indep X $ \par
\ghl{Interprétation \textcolor{BrickRed}{quantitative directe} :} la variable d'intérêt est $Y$ et $Y^{*}$ n'est qu'une variable latente qui n'a pas forcément de sens quantitatif précis. Le paramètre d'intérêt est l'effet marginal de $X_{k}$ sur la variable d'intérêt, ici $Y$. Les coefficients de $\beta_{0}$ concernant $Y^{*}$ on ne peut donc pas directement interpréter quantitativement ces derniers sur la variable d'intérêt $Y$. De plus, l'effet marginal de $X_{k}$ sur $Y$ est différent de $\beta_{0k}$ : c'est pour cela qu'on ne peut avoir d'interprétation quantitative des coefficients de $\beta_{0}$ ! \par
\ghl{Interprétation \textcolor{ForestGreen}{qualitative} :} en revanche le signe de l'effet marginal de $X_{k}$ sur $Y$, i.e. effet positif ou négatif sur $P(Y=1|X)$, est donné par le signe de $\beta_{0k}$. \par
On peut en revanche comparer quantitativement le ratio des effets marginaux des variables i et j qui vaut $\widehat{\beta_{i}}/\widehat{\beta_{j}}$. \textit{L'effet sur la proba d'être ... de la variable i est <quantitativement> ... que l'effet de la variable j $\Longleftrightarrow$ regarder le rapport $\widehat{\beta_{i}}/\widehat{\beta_{j}}$ }.

\bigbreak
\noindent $\hookrightarrow$ \underline{\textbf{Modèle de censure / Tobit1 :}} \\
\textcolor{blue}{1 seul mécanisme détermine la valeur de $Y$ et si on observe la variable d'intérêt ou non. } Deux cas sont à distinguer :\par
\bigbreak
$\Rightarrow$ \textbf{\textcolor{ForestGreen}{Données censurées}} : la variable d'intérêt est $Y^{*}$ qui peut ne pas être observée au dessous d'un seuil causant un problème de censure. Le paramètre d'intérêt est l'effet marginal de $X_{k}$ sur la variable d'intérêt $Y^{*}$, qui vaut bien $\beta_{0k}$ lorsque $X_{k}$ apparaît simplement dans le modèle linéaire de $Y^{*}$. Ainsi, la variable $Y^{*}$ ayant un sens quantitatif et malgré la censure liée aux problèmes d'observation on peut bien interpréter quantitativement $\beta_{0}$ sur la variable d'intérêt.
\bigbreak
$\Rightarrow$ \textbf{\textcolor{BrickRed}{Solution en coin}} : la variable d'intérêt est bien $Y$ alors que la variable $Y^{*}$ est une variable latente potentiellement dépourvue de sens quantitatif. Typiquement un pb d'optimisation du consommateur où $Y^{*}$ mesure l'utilité optimale (en nombre de biens) de consommation d'un bien donné : donc potentiellement négatif. Et Y représente le nombre d'unités effectivement consommées. Les coefficients de $\beta_{0}$ concernant $Y^{*}$ qui n'as pas de sens quantitatif précis : on ne peut pas interpréter quantitativement les coefficients de $\beta_{0}$ sur la variable d'intérêt $Y$. Les paramètres d'intérêt sont les effets marginaux : le total $\frac{\partial E[Y|X=x]}{\partial x_{k}}$ (marge extensive et intensive) et $\frac{\partial E[Y|Y>0,X=x]}{\partial x_{k}}$ (marge intensive seulement). Ces paramètres sont tous les deux différents de $\beta_{0k}$ ce qui explique le manque d'interprétation quantitative des coefficients de $\beta_{0}$.
\bigbreak
\noindent $\hookrightarrow$ \underline{\textbf{Modèle de sélection / Tobit2 :}} \\
\textcolor{blue}{Ici on a bien deux processus différents : un qui détermine $Y^{*}$ \textbf{et un autre} qui détermine si on observe cette valeur ou non i.e. modèle sur D.}\par 
\ghl{Interprétation \textcolor{ForestGreen}{quantitative directe} :} il y a un problème d'observation des données, on observe $Y=D.Y^{*}$ mais la variable d'intérêt est bien $Y^{*}$ (\ghl{variable potentielle qui existe pour tous} \ghl{les \textit{individus}}). Par conséquent $Y^{*}$ suivant un modèle linéaire, les paramètres d'intérêts sont les effets marginaux des variables explicatives sur la variable d'intérêt $Y^{*}$ et les coefficients de $\beta_{0}$ sont toujours interprétables quantitativement.

\newpage





\end{document}